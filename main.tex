\documentclass{article}

\usepackage[english]{babel}

\usepackage{graphicx} % Required for inserting images
\usepackage{amsmath}
\usepackage{amssymb}
\usepackage{forest}
\usepackage{amsthm}
\usepackage{cancel}
\usepackage{tabularray}
\usepackage{xfrac}
\UseTblrLibrary{booktabs} % pro pěknější čáry
\UseTblrLibrary{siunitx}  % pro čísla s oddělovačem
\usepackage{xcolor}
\usepackage{tabularray}
\usepackage{xurl} % Enables flexible URL line breaking
\usepackage{hyperref} % For clickable links
\UseTblrLibrary{diagbox}


\theoremstyle{remark}
\newtheorem{remark}{Remark}

\theoremstyle{theorem}
\newtheorem{proposition}{Proposition}[section]

\newtheorem{theorem}{Theorem}[section]

\newtheorem{lemma}{Lemma}[section]

\theoremstyle{definition}
\newtheorem{definition}{Definition}[section]


\title{treti clanek}
\author{Adam Ulrich}
\date{November 2025}

\begin{document}

\maketitle

\section{Introduction}
\begin{itemize}
    \item V citacich ukazeme ze je potreba resit missing data
    \item pripomeneme strom a $c(k)$
\end{itemize}

Let an HS-Tree be built from $n$ data points. Then the expected value of the depth of a data point in the tree is given by (viz druhý článek, CITACE)

\begin{align}
c(n) &=
\begin{cases}
0  &  n = 0, \\
1+\dfrac{\sum_{k=0}^{n} \binom{n}{k}c(k)}{2^n } &  n = 1,2,3,\dots.
\end{cases}\label{eq:c'n_vyjadreno}
\end{align}

\section{Co je to crk -- lepsi verze}
\begin{itemize}
    \item nejprve zavedu problem, missing features v datech, chci pouzit nas HS tree, viz introduction, popis se ze to deje casto v datech a ZMINIT CLANEK BURDA NOVAK
    \item chci pouzit nase c(n) protoze je to fajn funkce dobre aproximovatelna
    \item mam problem kdyz nebyly missing data, proste jsme pouzili c(n), ale ted se nám kvuli Rkam exptected depth prodlouzi
    \item takze po kazdem deleni se uz nezapocitava +1 k depth ale nejake +a k depth
    \item a ted bychom chteli to $a$ takove, aby se celkova ocekavana depth nezmenila, to acko tedy vyrovna ten rozdil mezi tou depth c(k) mez missing a mezi (logicky vetsi) depth s missing.
\end{itemize}


\subsection{Jak teda $a$cko spocitat/odvodit?}

Počet $R$ v uzlu odpovídá počtu \emph{missing} features.  

Podobně jako v článku~\cite{ulrich2024} využijeme i zde analogii s mincemi pro vysvětlení
náhodného dělení prostoru v Half-Space Trees. Tato metafora se ukázala jako mimořádně
užitečná: každý prvek (data point) se při náhodném dělení chová podobně jako mince, která
padá vlevo nebo vpravo podle toho, na které straně náhodně zvolené hyperroviny leží.

HS-Tree tak můžeme interpretovat jako binární strom, kde se každý prvek v každém uzlu
náhodně rozhoduje pro levou či pravou větev. Pokud známe počet prvků v uzlu, pak hloubka
tohoto procesu závisí jen na tom, kolikrát se prvek rozdělí mezi větve---a tedy na počtu
možných výsledků těchto ``náhodných hodů''.  

Tato abstrakce nám dovoluje nahradit geometrické dělení jednoduchým modelem s mincemi:
\begin{itemize}
    \item H představuje data, která jednoznačně spadnou do levého half-space,
    \item T představuje data, která spadnou do pravého half-space,
    \item a zavádíme také stav R (\emph{edge}), který reprezentuje chybějící nebo
          neurčitou hodnotu: takový prvek musíme poslat do obou větví současně.
\end{itemize}

Mince jsou tedy jen zjednodušený, ale přesný model chování bodů v Half-Space Tree:
každý hod mincí zachycuje jeden krok náhodného dělení prostoru a umožňuje nám formálně
vypsat všechny možné výsledky prvního rozdělení. Na těchto kombinacích pak stavíme
výpočet očekávané hloubky.

\paragraph{Motivace k tabulce stromečků.}
V kořeni stromu hodíme všemi mincemi. Každá větev dostane odpovídající podmnožinu
mincí a celý proces se v uzlech opakuje: znovu se hodí všemi mincemi, které do něj
spadly. Strom pokračuje, dokud se mince nerozdělí natolik, že v listu zůstane pouze
jediná mince.

Očekávaná hloubka takového stromu je průměr přes všechny možné průběhy všech
hodů. To je však obrovské množství možností.

\paragraph{Proč nás zajímá pouze první hod?}
Protože se pokusíme najít funkci takovou, která by dala očekávanou hloubku, za předpokladu, že dál už missing data nebudou a pojede se podle $c(n)$ kde n je počet prvků v aktuálním podstromu.

Proto i Tabulka \ref{tab:stromecek_batch3_1R} když se podíváš zachycuje všechny možné výsledky \textbf{prvního rozdělení mincí}.

Teď chceme tedy najít nějaký parametr $a$ takový, který když přidáme ke každé větvi s missing features (R) tak to vyjde stejně jakobychom missing features neměli $c(n)$. To je vyhodne zejmena protoze $c(n)$ uz umime velmi dobre pocitat i aproximovat.

\subsection{Příklad s jedním R}
Zkusme si tedy příklad kde máme jednu missing feature a batch size 3. Tabulka \ref{tab:stromecek_batch3_1R} nam ukaze jak to dopadne pro vsechny moznosti. Pomoci ni zkusime spocitat tedy ocekavanou hloubku $c(3)$, ale pozor, uz budeme přičítat parametr $a$ jako cenu za první krok, dál už je to $c(k)$, kde $k$ je pocet prvku v dané větvi (WLOG vždy levé).


\begin{table}[ht]
\centering
\scriptsize
\setlength{\tabcolsep}{6pt}
\renewcommand{\arraystretch}{1.1}
\caption{Všechny možné výsledky prvního hodu pro tři mince s jedním~R.}
\label{tab:stromecek_batch3_1R}
\begin{tabular}{|p{0.22\linewidth}|p{0.22\linewidth}|p{0.22\linewidth}|p{0.22\linewidth}|}
\hline
\begin{minipage}[t]{\linewidth}\begin{verbatim}
{RHH}
 /    \
{RHH}  {R}
\end{verbatim}\end{minipage} &
\begin{minipage}[t]{\linewidth}\begin{verbatim}
{RHT}
 /    \
{RH}   {RT}
\end{verbatim}\end{minipage} &
\begin{minipage}[t]{\linewidth}\begin{verbatim}
{RTH}
 /    \
{RH}   {RT}
\end{verbatim}\end{minipage} &
\begin{minipage}[t]{\linewidth}\begin{verbatim}
{RTT}
 /    \
{R}    {RTT}
\end{verbatim}\end{minipage}
\\ \hline
\begin{minipage}[t]{\linewidth}\begin{verbatim}
{HRH}
 /    \
{RHH}  {R}
\end{verbatim}\end{minipage} &
\begin{minipage}[t]{\linewidth}\begin{verbatim}
{HRT}
 /    \
{RH}   {RT}
\end{verbatim}\end{minipage} &
\begin{minipage}[t]{\linewidth}\begin{verbatim}
{TRH}
 /    \
{RH}   {RT}
\end{verbatim}\end{minipage} &
\begin{minipage}[t]{\linewidth}\begin{verbatim}
{TRT}
 /    \
{R}    {RTT}
\end{verbatim}\end{minipage}
\\ \hline
\begin{minipage}[t]{\linewidth}\begin{verbatim}
{HHR}
 /    \
{RHH}  {R}
\end{verbatim}\end{minipage} &
\begin{minipage}[t]{\linewidth}\begin{verbatim}
{HTR}
 /    \
{RH}   {RT}
\end{verbatim}\end{minipage} &
\begin{minipage}[t]{\linewidth}\begin{verbatim}
{THR}
 /    \
{RH}   {RT}
\end{verbatim}\end{minipage} &
\begin{minipage}[t]{\linewidth}\begin{verbatim}
{TTR}
 /    \
{R}    {RTT}
\end{verbatim}\end{minipage}
\\ \hline
\end{tabular}
\end{table}



\begin{enumerate}
    \item nejprve přidáme stromečky s R, například pro batch 3, stačí si uvědomit že ocekavana hloubka je hloubka vsech kombinací, takze nahore vezmeme pocet prvku faktorial a podelime vzdy poctem R,poctem H a poctem T v tomto poradi, vynasobime a+c(pocet prvku v leve vetvi) - WLOG bereme jen levou vetev:
    $$c(3) = \frac{1}{12}\left(\frac{3!}{1!2!0!}(a+c(3))+\frac{3!}{1!0!2!}(a+c(1))+\frac{3!}{1!1!1!}(a+c(2))\right)$$
    
    \item vyjádříme $a$:
    $$a_{1,3} = c(3) - \left( \tfrac14 c(3) + \tfrac14 c(1) + \tfrac12 c(2) \right)$$

    \item vyjádříme obecně $a_{r,k}$ 

\end{enumerate}


Díky symetrii (WLOG můžeme předpokládat, že $R$ jsou vždy vlevo), pak:


\begin{enumerate}
\item
        \begin{align}
        c(3)
        &= \frac{1}{4}\Bigl(
            \binom{2}{0}(a_{1,3}+c(1))
            + \binom{2}{1}(a_{1,3}+c(2))
            + \binom{2}{2}(a_{1,3}+c(3))
        \Bigr)
    \end{align}



\item vyjádříme $a$
    \[
        a_{1,3}
        = c(3)
        -\left(
            \frac{1}{4}\binom{2}{0}c(1)
            + \frac{1}{4}\binom{2}{1}c(2)
            + \frac{1}{4}\binom{2}{2}c(3)
        \right).
    \]

\end{enumerate}

Obecně tedy:

\begin{align}
    c(k) 
        &=1+\dfrac{\sum_{k=0}^{n} \binom{n}{k}(a_{r,k}+c(k))}{2^n }=\\
        &= \frac{\sum_{i=0}^{k-r} \binom{k-r}{i} (a_{r,k} + c(r+i))}
                {\sum_{j=0}^{k-r} \binom{k-r}{j}} = \\
        &= \frac{a_{r,k}\sum_{i=0}^{k-r} \binom{k-r}{i}  + \sum_{i=0}^{k-r} \binom{k-r}{i} c(r+i)}
                {\sum_{j=0}^{k-r} \binom{k-r}{j}} = \\
        &= a_{r,k}
           + \frac{\sum_{i=0}^{k-r} \binom{k-r}{i} \cdot c(r+i)}
                  {2^{\,k-r}}.
\end{align}
Odtud dostáváme
\begin{align}
    a_{r,k} = c(k) - \frac{\sum_{i=0}^{k-r} \binom{k-r}{i} \cdot c(r+i)}
                  {2^{\,k-r}}
\end{align}
Označme si menšitel rozdílu (subtrahend) jako $c(r,k)$:
\begin{definition}
\label{definition:c(r,k)}
Pro dvě celá čísla r,k větší nebo rovno nula $0\le r \le k$ definujme
\begin{align}
    c(r,k) = \frac{\sum_{i=0}^{k-r} \binom{k-r}{i} \cdot c(r+i)}{2^{\,k-r}}.
\end{align}    
\end{definition}


Můžeme zkusit par hodnot do Tabulky \ref{tab:cprime_values} (pro $c(n)$) a do Tabulky \ref{tab:crk_values} (pro $c(r,k)$).
\begin{table}[htp]
\centering
\caption{Sample values of $c(n)$ with decimal and fractional approximations}
\begin{tblr}{
  colspec = {r S[table-format=2.5] S},
  column{2-3} = {mode=math},
  row{1} = {guard,font=\bfseries},
}
{n} & {Decimal} & {Fraction} \\
\midrule
0  & 0        & \sfrac{0}{1} \\
1  & 2        & \sfrac{2}{1} \\
2  & 2.667    & \sfrac{8}{3} \\
3  & 3.14286  & \sfrac{22}{7} \\
4  & 3.505    & \sfrac{368}{105} \\
5  & 3.794    & \sfrac{2\,470}{651} \\
6  & 4.0348   & \sfrac{7\,880}{1\,953} \\
7  & 4.240    & \sfrac{150\,266}{35\,433} \\
8  & 4.4210   & \sfrac{13\,315\,424}{3\,011\,805} \\
9  & 4.581    & \sfrac{2\,350\,261\,538}{513\,010\,785} \\
10 & 4.725    & \sfrac{1\,777\,792\,792}{376\,207\,909} \\
11 & 4.856    & \sfrac{340\,013\,628\,538}{70\,008\,871\,793} \\
\vdots & \vdots & \vdots \\
1024 & 11.335    & \vdots \\
2048 & 12.3331   & \vdots \\
4096 & 13.3329   & \vdots \\
8192 & 14.3328   & \vdots \\
%\bottomrule
\end{tblr}
\label{tab:cprime_values}
\end{table}

\begin{table}[ht]
\centering
\caption{$c(r,k)$ values}
\label{tab:crk_values}

\begin{tblr}{
  colspec = {c c c c c c c},
  hline{2} = {1pt},
  vline{2} = {1pt},
  row{1} = {font=\bfseries},
  row{2-Z} = {rowsep=6pt},
  column{1} = {font=\bfseries},
}

% Horní levá buňka: r (řádky), k (sloupce)
\diagbox{r}{k}
& 0 & 1 & 2 & 3 & 4 & 5 \\

0 &
$\mathbf{\frac{0}{1}}$ &
$\frac{1}{1}$ &
$\frac{5}{3}$ &
$\frac{15}{7}$ &
$\frac{263}{105}$ &
$\frac{1819}{651}$ \\

1 &
-- &
$\mathbf{\frac{2}{1}}$ &
$\frac{7}{3}$ &
$\frac{55}{21}$ &
$\frac{43}{15}$ &
$\frac{10037}{3255}$ \\

2 &
-- & -- &
$\mathbf{\frac{8}{3}}$ &
$\frac{61}{21}$ &
$\frac{109}{35}$ &
$\frac{3581}{1085}$ \\

3 &
-- & -- & -- &
$\mathbf{\frac{22}{7}}$ &
$\frac{349}{105}$ &
$\frac{3783}{1085}$ \\

4 &
-- & -- & -- & -- &
$\mathbf{\frac{368}{105}}$ &
$\frac{1697}{465}$ \\

5 &
-- & -- & -- & -- & -- &
$\mathbf{\frac{2470}{651}}$ \\

\end{tblr}
\end{table}


\subsection{Interesting consequences of the definitions}

The mapping $c(r,k)$ satisfies the following boundary identities:

\begin{lemma}
\label{lemma:crk}
The following relations hold for all $k \ge 0$:
\begin{enumerate}
    \item $c(k,k) = c(k)$,
    \item $c(0,k) = c(k) - 1$, for all $k > 0$.
\end{enumerate}
\end{lemma}

\begin{proof}

\begin{align*}
c(k,k)&= \frac{\sum_{i'=0}^{0} \binom{0}{i'}\,c(k+i')}{2^{0}}
     = \binom{0}{0}\,c(k)
     = c(k).\\
c(n) &= 1+\frac{\sum_{k=0}^{n} \binom{n}{k} c(k)}{2^n} =1+c(0,k).
\end{align*}

\end{proof}


\section{Defining \(c(r,k)\) independently of \(c(k)\)}
While \(c(r,k)\) can be computed directly from the values \(c(k)\) as illustrated 
above, this approach is unnecessarily cumbersome once one recalls the structure 
of \(c(k)\).  We therefore seek an intrinsic characterisation of \(c(r,k)\) that 
does not rely on \(c(k)\).

\subsection{\( c(r, k) \) as a solution to a system of linear equations}

\begin{theorem}
\label{theorem:3equations_of_crk}
For integers \(k>r\ge 0\), the function \(c(r,k)\) satisfies:
\begin{align}
    c(0,0) &= 0, \label{eq:c00}\\
    c(k,k)-c(0,k) &= 1, \label{eq:ckk-c0k}\\
    c(r,k) &= \frac{c(r,k-1) + c(r+1,k)}{2}. \label{eq:crk_to_proof}
\end{align}
\end{theorem}

\begin{proof}
Equalities~\eqref{eq:c00} and~\eqref{eq:ckk-c0k} follow directly from 
Lemma~\ref{lemma:crk}.  
To prove~\eqref{eq:crk_to_proof}, we expand the right–hand side using the 
definition of \(c(r,k)\) accordingly:
\begin{multline*}
\frac{c(r,k-1)+c(r+1,k)}{2}=\\
=\frac{1}{2}\left(\frac{\sum_{i=0}^{k-1-r}\binom{k-1-r}{i}c(r+i)}{2^{k-1-r}}+\frac{\sum_{i=0}^{k-r-1}\binom{k-r-1}{i}c(r+1+i)}{2^{k-r-1}}\right)=\\
 =\frac{1}{2^{k-r}}\left(\textstyle\binom{k-1-r}{0}\,c(r)
 +\sum_{i=1}^{k-r-1}\left(\binom{k-1-r}{i}+\binom{k-r-1}{i-1}\right)c(r+i)
 +\binom{k-r-1}{k-r-1}\,c(k)\right) =\\
=\frac{1}{2^{k-r}}\left(\binom{k-r}{0}\,c(r)+\sum_{i=1}^{k-r-1}\,\binom{k-r}{i}\,c(r+i)+\binom{k-r}{k-r}\,c(k)\right)=\\
=\frac{1}{2^{k-r}}\sum_{j=0}^{k-r}c(r+j)\binom{k-r}{j}=c(r,k).
\end{multline*}

\end{proof}


\subsection{Uniqueness of the coefficients}

In this subsection we show that the values \(c(r,k)\) are uniquely determined.

Mějme sadu rovnic
\begin{align*}
    c_{0,0}&=0\\
    c_{k,k}-c_{0,k}&=1\\
    c_{r,k}&=\frac{c_{r,k-1}+c_{r+1,k}}{2}
\end{align*}
Tato soustava má jediné řešení, dokážu:
$c_{0,0}=0$, to je jednoznačně dané.
Pak indukcí: $c_{r,k-1}\rightarrow c_{r,k}$
Pod podmínkou $0\le r\le k$, $r,k\in \mathbb{N}$ přepíšu soustavu do tvaru:
\begin{align*}
    2c_{r,k}-c_{r+1,k}&=c_{r,k-1}\\
    c_{k,k}-c_{0,k}&=1
\end{align*}
A zapíšu do matice

\[
\left[
\begin{array}{ccccc|c}
    2 & -1 & 0 & \cdots & 0 & c_{0,k-1} \\
    0 & 2 & -1 & \cdots & 0 & c_{1,k-1} \\
    \vdots & \vdots & \vdots & \ddots & \vdots & \vdots \\
    0 & 0 & \cdots & 2 & -1 & c_{k-1,k-1} \\
    -1 & 0 & \cdots & 0 & 1 & 1
\end{array}
\right]
\]

Snadno z Laplaceova rozvoje podle posledního řádku plyne, že determint matice soustavy je nenulový, resp.
\begin{align*}
  \begin{vmatrix}
    2 & -1 & 0 & \cdots & 0& 0 \\
    0 & 2 & -1 & \cdots & 0& 0 \\
    0 & 0 & 2 & \cdots & 0& 0 \\
    \vdots & \vdots & \vdots & \ddots & \vdots &\vdots\\
    0 & 0 & 0&\cdots & 2 & -1  \\
    -1 & 0 & 0&\cdots & 0 & 1 
  \end{vmatrix}&=2^k-1\\
\end{align*}

% Determinant matice $D$ je nenulový, protože:
% Přímo z definice jediné nenulové permutace jsou:
% \begin{enumerate}
%     \item $k+1$ řádků od $2$ do $k$, dává $2^k$
%     \item permutace $(-1)^{k+1}$
%     Ale musime ještě správné znaménko, takže počet inverzí je počet prvků před jedničkou, tj. $k$. Pak počet permutací je $2^k+(-1)^{2k+1}=2^k-1$
% \end{enumerate}
% Tím pádem přímo z definice je determinant nenulový $\rightarrow$ soustava má 1 řešení.
% Známe tedy $c_{r,k-1}$ pro $0,....k-1$.
% Z toho tedy jednoznačně určíme $c(r,k)$ pro $r=0,....k$.

\begin{remark}
It is possible to compute column $k+1$ from column $k$.

We now express each value $c(r,k)$, for $0 \le r \le k$, solely in terms of the
preceding column $c(r,k-1)$.

\begin{enumerate}
    \item Iterating the $k$–recurrence.

Starting from
\[
c(k+1,n)=2c(k,n)-c(k,n-1),
\]
we expand successively:
\begin{align*}
c(1,n) &= 2c(0,n)-c(0,n-1),\\
c(2,n) &= 2c(1,n)-c(1,n-1)
       = 2^{2}c(0,n)-2c(0,n-1)-c(1,n-1),\\
c(3,n) &= 2c(2,n)-c(2,n-1)
       = 2^{3}c(0,n)-2^{2}c(0,n-1)-2c(1,n-1)-c(2,n-1),
\end{align*}
and so on.  
By induction, for every $k\ge1$,
\begin{align}
c(k,n)
 = 2^{k}c(0,n)
   -\sum_{j=0}^{k-1}2^{\,k-1-j}\,c(j,n-1).
\label{eq:ckn_iterated}
\end{align}
The case $k=0$ is trivial: $c(0,n)=c(0,n)$.

\item Closing the formula using the boundary condition.
Setting $k=n$ in \eqref{eq:ckn_iterated} and using the boundary relation
$c(n,n)=c(0,n)+1$, we obtain
\[
2^{n}c(0,n)-\sum_{j=0}^{n-1}2^{\,n-1-j}\,c(j,n-1)=c(0,n)+1.
\]
Solving for $c(0,n)$ gives
\begin{align}
c(0,n)
 = \frac{
        1 + \displaystyle\sum_{j=0}^{n-1}2^{\,n-1-j}c(j,n-1)
      }{2^{n}-1}.
\label{eq:c0n_closed}
\end{align}

\item Column transformation.

Define the weighted sum
\[
S_n := \sum_{j=0}^{n-1}2^{\,n-1-j}\,c(j,n-1).
\]
Then by \eqref{eq:c0n_closed} and \eqref{eq:ckn_iterated},
\begin{align*}
c(0,n) &= \frac{1+S_n}{2^{n}-1},\\[2pt]
c(k,n) &= 2^{k}c(0,n)
         -\sum_{j=0}^{k-1}2^{\,k-1-j}\,c(j,n-1),
         \qquad k=1,\dots,n.
\end{align*}

\item Example.

For $n=2$ let
\[
\bigl(c(0,2),c(1,2),c(2,2)\bigr)
   =\left(\frac{5}{3},\frac{7}{3},\frac{8}{3}\right).
\]
Then
\[
S_3
 = 2^{2}c(0,2)+2^{1}c(1,2)+2^{0}c(2,2)
 = 14,
\]
and hence
\[
c(0,3)=\frac{1+S_3}{2^{3}-1}=\frac{15}{7}.
\]
The remaining values follow from the recursion:
\begin{align*}
c(1,3)
 &=2c(0,3)-c(0,2)
 =\frac{55}{21},\\[2pt]
c(2,3)
 &=4c(0,3)-\bigl(2c(0,2)+c(1,2)\bigr)
 =\frac{61}{21},\\[2pt]
c(3,3)
 &=8c(0,3)-\bigl(4c(0,2)+2c(1,2)+c(2,2)\bigr)
 =\frac{22}{7}.
\end{align*}
Thus the column for $n=3$ is
\[
\bigl(c(0,3),c(1,3),c(2,3),c(3,3)\bigr)
=\left(\frac{15}{7},\frac{55}{21},\frac{61}{21},\frac{22}{7}\right).
\]
\end{enumerate}

\end{remark}



% stara ceska verze:
% \paragraph{Cíl.}
% Chceme vyjádřit \(c(k,n)\) (pro \(0\le k\le n\)) jen pomocí sloupce \(n{-}1\),
% tj. hodnot \(c(j,n{-}1)\) pro \(0\le j\le n{-}1\).

% \paragraph{Krok 1: rozvinutí rekurence v~$k$.}

% \paragraph{Iterační rozvinutí rekurence.}
% Vycházíme z~rekurence
% \[
% c(k{+}1,n)=2c(k,n)-c(k,n{-}1).
% \]

% Postupným dosazováním dostaneme:

% \[
% \begin{aligned}
% c(1,n) &= 2c(0,n)-c(0,n-1), \\
% c(2,n) &= 2c(1,n)-c(1,n-1) \\
%        &= 2\!\bigl(2c(0,n)-c(0,n-1)\bigr)-c(1,n-1) \\
%        &= 2^{2}c(0,n)-2c(0,n-1)-c(1,n-1), \\
% c(3,n) &= 2c(2,n)-c(2,n-1) \\
%        &= 2\!\left(2^{2}c(0,n)-2c(0,n-1)-c(1,n-1)\right)-c(2,n-1) \\
%        &= 2^{3}c(0,n)-2^{2}c(0,n-1)-2c(1,n-1)-c(2,n-1), \\
% &\qquad\vdots \\
% c(k,n) &= 2^{k}c(0,n)
%         \;-\;\Bigl(2^{k-1}c(0,n-1)+2^{k-2}c(1,n-1)+\cdots+2^{0}c(k-1,n-1)\Bigr).
% \end{aligned}
% \]

% Kompaktně:
% \[
% \boxed{
% c(k,n)
% =
% 2^{k}c(0,n)
% -\sum_{j=0}^{k-1}2^{\,k-1-j}\,c(j,n-1).
% }
% \]


% Z \(c(k{+}1,n)=2c(k,n)-c(k,n{-}1)\) iterací dostaneme
% \[
% c(k,n)=2^{k}\,c(0,n)\;-\;\sum_{j=0}^{k-1} 2^{\,k-1-j}\,c(j,n-1),\qquad k\ge 1,
% \]
% a triviálně \(c(0,n)=c(0,n)\).

% \paragraph{Krok 2: uzavření pomocí okraje.}
% Pro \(k=n\) a okraj \(c(n,n)=c(0,n)+1\) platí
% \[
% 2^{n}c(0,n)-\sum_{j=0}^{n-1}2^{\,n-1-j}c(j,n-1)=c(0,n)+1,
% \]
% tedy
% \[
% c(0,n)=\frac{1+\displaystyle\sum_{j=0}^{n-1}2^{\,n-1-j}\,c(j,n-1)}{2^{n}-1}.
% \]

% Nejprve spočti vážený součet
% \[
% S_n:=\sum_{j=0}^{n-1}2^{\,n-1-j}\,c(j,n-1).
% \]
% Pak
% \[
% \boxed{\;
% c(0,n)=\frac{1+S_n}{2^{n}-1},\qquad
% c(k,n)=2^{k}c(0,n)-\sum_{j=0}^{k-1}2^{\,k-1-j}\,c(j,n-1)\ \ (k=1,\dots,n).
% \;}
% \]

% \paragraph{Mini-příklad}
% \begin{enumerate}
%     \item Máme sloupec pro \(n=2\):
%     \[
%     (c(0,2),c(1,2),c(2,2))=\left(\frac{5}{3},\frac{7}{3},\frac{8}{3}\right).
%     \]

%     \item Spočítáme vážený součet
%     \[
%     S_3
%     =2^2c(0,2)+2^1c(1,2)+2^0c(2,2)
%     =\frac{20}{3}+\frac{14}{3}+\frac{8}{3}
%     =14.
%     \]

%     \item Získáme
%     \[
%     c(0,3)=\frac{1+S_3}{2^3-1}
%     =\frac{15}{7}.
%     \]

%     \item Další hodnoty dopočítáme vzorcem
%     \[
%     c(k,3)=2^k c(0,3)-\sum_{j=0}^{k-1}2^{k-1-j}c(j,2).
%     \]

%     \begin{enumerate}
%         \item Pro \(k=1\):
%         \[
%         c(1,3)=2c(0,3)-c(0,2)
%         =\frac{30}{7}-\frac{5}{3}
%         =\frac{55}{21}.
%         \]

%         \item Pro \(k=2\):
%         \[
%         c(2,3)=4c(0,3)-\left(2c(0,2)+c(1,2)\right)
%         =\frac{60}{7}-\frac{17}{3}
%         =\frac{61}{21}.
%         \]

%         \item Pro \(k=3\):
%         \[
%         c(3,3)=8c(0,3)-\left(4c(0,2)+2c(1,2)+c(2,2)\right)
%         =\frac{120}{7}-14
%         =\frac{22}{7}.
%         \]
%     \end{enumerate}

%     \item Výsledný sloupec pro \(n=3\) je
%     \[
%     \bigl(c(0,3),c(1,3),c(2,3),c(3,3)\bigr)
%     =
%     \left(\frac{15}{7},\frac{55}{21},\frac{61}{21},\frac{22}{7}\right).
%     \]
% \end{enumerate}


\section{Alternative definition}

\subsection{Theory}
\url{https://physics.bme.hu/sites/physics.bme.hu/files/users/BMETE15AF53_kov/Kreyszig%20-%20Introductory%20Functional%20Analysis%20with%20Applications%20(1).pdf} . 

The definitions used in this section follow the terminology of Kreyszig~\cite{kreyszig1991introductory}.


% \begin{definition}[The sequence space $\ell^{p}$]
% \label{kreyszig1.2-3}
% Let $p \ge 1$ be a fixed real number.
% The space $\ell^{p}$ consists of all sequences 
% \begin{align}
% x = (\xi_j) = (\xi_1, \xi_2, \ldots)    
% \end{align}
% of real or complex numbers for which the series
% \begin{align}
% |\xi_1|^{p} + |\xi_2|^{p} + \cdots    
% \end{align}
% is convergent.  
% Equivalently,
% \begin{align}
% \sum_{j=1}^{\infty} |\xi_j|^{p} < \infty,
% \qquad (p \ge 1\ \text{fixed})\label{eq:l2iflesstheninfty}.
% \end{align}

% The metric on $\ell^{p}$ is given by
% \[
% d(x,y)
%  = \left( 
%         \sum_{j=1}^{\infty} |\xi_j - \eta_j|^{p}
%    \right)^{1/p},
% \]
% where $y = (\eta_j)$ and the series $\sum |\eta_j|^{p}$ is also finite.

% \end{definition}
% The space $\ell^{p}$ introduced earlier admits a natural norm structure,
% which turns it into a fundamental example of a Banach space.  


% \begin{definition}[Norm on $\ell^{p}$]
% The norm on $\ell^{p}$ is defined by

% \begin{align}
% \|x\| = \left( \sum_{j=1}^{\infty} |\xi_j|^{p} \right)^{1/p}\label{eq:lpnorm},
% \end{align}
% where $x = (\xi_j)$ is a sequence in $\ell^{p}$.

% This norm induces the metric introduced previously in 1.2--3, since
% \[
% d(x,y) = \|x - y\|
%        = \left( \sum_{j=1}^{\infty} |\xi_j - \eta_j|^{p} \right)^{1/p},
% \]
% for $y = (\eta_j)$.  
% Kreyszig also established the completeness of $\ell^{p}$ with respect to this metric (see section 1.5--4. in \cite{kreyszig1991introductory}).
% \end{definition}

% The concept of an inner product is central to the study of Hilbert spaces.
% We follow the exposition of Kreyszig~\cite[Section~3.1]{kreyszig1991introductory}.

% \begin{definition}[Inner product space and Hilbert space]
% Let $X$ be a vector space over a scalar field $\mathbb{K}$ (either $\mathbb{R}$ or $\mathbb{C}$).
% An \emph{inner product} on $X$ is a mapping 
% \[
% \langle \cdot , \cdot \rangle : X \times X \to \mathbb{K},
% \]
% which associates to each pair of vectors $x,y \in X$ a scalar $\langle x , y\rangle$
% and satisfies the following properties for all vectors $x,y,z \in X$ and all scalars $\alpha$:

% \begin{align*}
% \text{(IP1)}\quad &\langle x+y, z \rangle = \langle x, z \rangle + \langle y, z \rangle, \\
% \text{(IP2)}\quad &\langle \alpha x, y \rangle = \alpha \langle x, y \rangle, \\
% \text{(IP3)}\quad &\langle x, y \rangle = \overline{\langle y, x \rangle}, \\
% \text{(IP4)}\quad &\langle x, x \rangle \ge 0, 
% \qquad 
% \langle x, x \rangle = 0 \ \Longleftrightarrow\ x = 0.
% \end{align*}

% A vector space equipped with an inner product is called an \emph{inner product space}
% (or pre-Hilbert space).  
% If the space is complete with respect to the metric induced by the inner product,
% it is called a \emph{Hilbert space}.  
% The corresponding norm is given by
% \begin{equation}
% \|x\| = \sqrt{\langle x, x \rangle}.
% \end{equation}

% Further details can be found in Kreyszig~\cite[Section~3.1]{kreyszig1991introductory}.
% \end{definition}

% Precisely, from now on, we shall mainly operate in the Hilbert sequence
% space $\ell^{2}$. This space provides a concrete and well–understood model of a
% Hilbert space and is a standard setting for the analysis of operators on
% infinite-dimensional spaces. We follow the presentation of 
% Kreyszig~\cite[Section~3.1--6]{kreyszig1991introductory}.

\begin{definition}[Hilbert sequence space $\ell^{2}$]
The space $\ell^{2}$ consists of all sequences 
$\mathsf{x} = (\xi_j) = (\xi_1, \xi_2, \ldots)$ of real or complex numbers for which
\begin{align}
\sum_{j=1}^{\infty} |\xi_j|^{2} < \infty \label{eq:l2iflesstheninfty}
\end{align}
that is, the series of squares is convergent.  

It is equipped with the inner product
\[
\langle x , y \rangle 
  = \sum_{j=1}^{\infty} \xi_j \,\overline{\eta_j},
\]
where $x = (\xi_j)$ and $y = (\eta_j)$ belong to $\ell^{2}$.  
The convergence of this series follows from the Cauchy--Schwarz inequality and
the assumption that $x,y\in \ell^{2}$.  

The induced norm is given by
\[
\|x\| = \langle x , x \rangle^{1/2}
      = \left( \sum_{j=1}^{\infty} |\xi_j|^{2} \right)^{1/2}.
\]

The completeness of $\ell^{2}$ with respect to this norm was established in Kreyszig's
Section~1.5--4, so $\ell^{2}$ forms a prototypical example of a Hilbert space.
\end{definition}


\subsection{Defining $c(r,k)$ in Hilbert space}

In the sequel we work entirely inside the Hilbert sequence space $\ell^{2}$, 
so vectors are understood as infinite sequences.  
In particular, a polynomial will be represented not by its symbolic form
$a_0 + a_1 x + \cdots + a_n x^n$, but by the corresponding coefficient 
sequence $(a_0, a_1, \ldots, a_n, 0, 0, 0, \dots)$, whose zeros extend to infinity.  
This identification is standard when polynomials are viewed as elements of a 
Hilbert space of sequences. However, we shall stick to the symbolic form notation.

\begin{definition}[Multiplication vector $v_c$]
Define the vector
\[
v_c = \bigl(0,\, 1,\,-\tfrac{1}{3},\,\tfrac{1}{7},\,\ldots,\,\tfrac{(-1)^{i+1}}{2^i-1},\,\ldots\bigr).
\]
\end{definition}

\begin{lemma}
    Multiplication vector $v_c$ satisfies all conditions for it to be part of $\ell^{2}$.
\end{lemma}

\begin{proof}
We recall the condition for item to be in $\ell^{2}$:
\begin{align}
\sum_{j=1}^{\infty} |\xi_j|^{2} < \infty. \label{eq:l2iflesstheninfty}
\end{align}
Indeed, for $\xi_j = \dfrac{(-1)^{j+1}}{2^j-1}$ we have
\[
|\xi_j|^{2} = \frac{1}{(2^j-1)^2} \le \frac{1}{4^j}.
\]
Thus,
\[
\sum_{j=1}^{\infty} |\xi_j|^{2} \le \sum_{j=1}^{\infty} \frac{1}{4^j},
\]
which is a convergent geometric series with ratio $\tfrac{1}{4}$. Therefore, $v_c \in \ell^{2}$.
\end{proof}

\begin{definition}[Infinite polynomial $P_{r,k}$]
For $k=0,1\dots$ and $r=0,1 \dots k$ we define the polynomial $P_{r,k}(x)$
\[
P_{r,k}(x) = (1+2x)^{r}(1+x)^{\,k-r}.
\]
In the Hilbert sequence space $\ell^{2}$ we identify $P_{r,n}$ with its
coefficient sequence
\[
P_{r,k} \;\equiv\; (\alpha_0, \alpha_1, \alpha_2, \ldots),
\]
where $\alpha_j$ is the coefficient of $x^{j}$ and all remaining entries are
$0$.  
Thus each $P_{r,k}$ is viewed as an element of $\ell^{2}$ and inner products
such as $\langle P_{r,k}, v^{(k)} \rangle$ are understood in the standard 
Hilbert space sense.
\end{definition}

We now show that the sequence $P_{r,n}$ belongs to the space $\ell^{2}$ by 
verifying that it satisfies the defining condition~\eqref{eq:l2iflesstheninfty}.

\begin{proof}
By Definition \eqref{eq:l2iflesstheninfty} we identify the polynomial
$P_{r,k}(x)$ with its coefficient sequence
\[
P_{r,k} \;\equiv\; (\alpha_0, \alpha_1, \ldots, \alpha_k, 0, 0, \dots).
\]
In particular, $\alpha_j = 0$ for all $j > k$.  
Therefore the series in~\eqref{eq:l2iflesstheninfty} reduces to a finite sum,
\[
\sum_{j=0}^{\infty} |\alpha_j|^{2}
  = \sum_{j=0}^{k} |\alpha_j|^{2} < \infty.
\]
Hence $P_{r,k}$ satisfies the defining condition~\eqref{eq:l2iflesstheninfty},
and we conclude that $P_{r,k} \in \ell^{2}$.
\end{proof}

We can directly show that the polynomials $P_{r,k}$ satisfy the same three–term
recurrence as $c(r,k)$:
\begin{align}
2P_{r,k}(x) = P_{r,k-1}(x) + P_{r+1,k}(x), \qquad 0 \le r < k\label{eq:2prk=prk+prk},    
\end{align}


where $P_{r,k}(x) = (1+2x)^{r}(1+x)^{\,k-r}$.

\begin{proof}
By definition,
\[
P_{r,k-1}(x) = (1+2x)^{r}(1+x)^{k-1-r},
\qquad
P_{r+1,k}(x) = (1+2x)^{r+1}(1+x)^{k-1-r}.
\]
Hence
\begin{align*}
P_{r,k-1}(x) + P_{r+1,k}(x)
&= (1+2x)^{r}(1+x)^{k-1-r}\bigl(1 + (1+2x)\bigr) \\
&= (1+2x)^{r}(1+x)^{k-1-r}(2+2x) \\
&= 2(1+2x)^{r}(1+x)^{k-r} \\
&= 2P_{r,k}(x),
\end{align*}
which is the claimed identity.
\end{proof}


\begin{theorem}
For integers $0 \le r \le k$, the coefficient $c(r,k)$ can be expressed as the
inner product of the multiplication vector $v_c$ and the polynomial $P_{r,k}$ in
the Hilbert sequence space $\ell^{2}$:
\[
c(r,k) = \langle v_c, P_{r,k} \rangle.
\]
\end{theorem}

% TODO: TOTO CELE SMAZAT
% Řekneme že Prk vychází z pascala ale je to trochu říznuté:


% Klasický binomický rozvoj dává řádky Pascalova trojúhelníku:
% \[
% (1+x)^n=\sum_{j=0}^n \binom{n}{j}x^j.
% \]
% Pro \(n=4\) (což je matice 5x5) dostaneme koeficienty \([1,4,6,4,1]\).
% To je \emph{první (horní) řádek} naší matice \(5\times5\). Další řádky se u nás ale liší

% \[
% \begin{bmatrix}
% \textbf{1} & \textbf{4} & \textbf{6} & \textbf{4} & \textbf{1} \\
% 1 & 5 & 9 & 7 & 2 \\
% 1 & 6 & 13 & 12 & 4 \\
% 1 & 7 & 18 & 19 & 8 \\
% 1 & 8 & 24 & 32 & 16 \\
% \end{bmatrix}
% \]

% Pro úplnost Tabulka \ref{tab:pascal_matrices} obsahuje predchozi matice.

% \begin{table}[h]
% \centering
% \renewcommand{\arraystretch}{1.2}
% \setlength{\tabcolsep}{8pt}
% \begin{tabular}{c c c c}
% \toprule
% $1\times 1$ & $2\times 2$ & $3\times 3$ & $4\times 4$ \\
% \midrule
% $\begin{bmatrix}
% 1
% \end{bmatrix}$
% &
% $\begin{bmatrix}
% 1 & 1 \\
% 1 & 2
% \end{bmatrix}$
% &
% $\begin{bmatrix}
% 1 & 2 & 1 \\
% 1 & 3 & 2 \\
% 1 & 4 & 4
% \end{bmatrix}$
% &
% $\begin{bmatrix}
% 1 & 3 & 3 & 1 \\
% 1 & 4 & 5 & 2 \\
% 1 & 5 & 8 & 4 \\
% 1 & 6 & 12 & 8
% \end{bmatrix}$ \\
% \bottomrule
% \end{tabular}
% \caption{První čtyři matice (od $1\times1$ do $4\times4$).}
% \label{tab:pascal_matrices}
% \end{table}

% Tedy řádek s $r$ kroky typu $R$ je reprezentován polynomem
% \[
% P_{r,n}(x)=(1+2x)^r(1+x)^{\,n-r}.
% \]

% \paragraph{Mini–příklad.}
% Pro $n=4$ a $r=1$ máme
% \[
% P_{1,4}(x)=(1+2x)(1+x)^3.
% \]
% Rozviňme $(1+x)^3=1+3x+3x^2+x^3$, takže
% \[
% P_{1,4}(x)=(1+2x)(1+3x+3x^2+x^3).
% \]
% Po roznásobení dostáváme
% \[
% P_{1,4}(x)=1+5x+9x^2+7x^3+2x^4.
% \]


% \noindent
% Zatimco klasický pascalovský řádek pro $n=4$ je
% \[
% (1+x)^4=1+4x+6x^2+4x^3+x^4.
% \]


% \paragraph{Zavedeme váhový vektor.}
% Definuj
% \[
% \mathbf{v}^{(k)}=\biggl(0,\;\frac{(-1)^{0}}{2^1-1},\;\frac{(-1)^{1}}{2^2-1},\;\dots,\;\frac{(-1)^{k-1}}{2^k-1}\biggr)
% =\Bigl(0,\;1,\;-\tfrac13,\;\tfrac17,\;-\tfrac1{15},\;\dots\frac{(-1)^{k-1}}{2^k-1}\Bigr).
% \]


% \paragraph{Příklad 1: \(c(0,3)\).}
% První řádek (\(r=0\)) matice \(4\times 4\) je
% \[
% \mathbf{M}_0 = (1,\,3,\,3,\,1).
% \]

% Respektive použijeme když použijeme přímo polynom $P_{r,k}$:
% \begin{align*}
% P_{r,n}(x)=(1+2x)^r(1+x)^{\,n-r}\\
% P_{0,3}(x) = (1+x)^3 = x^{3} + 3x^{2} + 3x + 1.    
% \end{align*}

% Mějme Euklidovský prostor. Skalární součin s váhovým vektorem \(\mathbf{v}^{(3)}\) dává
% \[
% c(0,3)=\langle\mathbf{M}_0, \mathbf{v}^{(3)}\rangle
% = 3\cdot 1 + 3\cdot\!\Bigl(-\tfrac{1}{3}\Bigr) + 1\cdot \tfrac{1}{7}
% = \tfrac{15}{7}.
% \]

% Z toho lze tedy videt
% \begin{theorem}
% Pro $k>r\ge0$ zobrazení $c(r,k)$ splňuje následující

% \[
% c(r,k) = \langle (0,\bigg(\frac{(-1)^{j-1}}{2^j-1}\bigg)^{\infty}_{j=1}), P_{r,k} \rangle.
% \]
% \end{theorem}




With $c(r,k)$ calculated this way, let us prove it satisfies all three equations proposed in Theorem \ref{theorem:3equations_of_crk}.



\begin{proof}
Let us prove the equations \eqref{eq:c00}, \eqref{eq:ckk-c0k},\eqref{eq:crk_to_proof} resectively.
\begin{enumerate}
    \item $c(0,0) = \langle v_c, P_{0,0}\rangle = (0,\,1,\,-\tfrac{1}{3},\,\tfrac{1}{7},\ldots) \cdot (1,0,0,\ldots) = 0$.

\item $c(k,k)-c(0,k) = 1$:

Recall that
\[
c(r,k) = \langle v_c, P_{r,k} \rangle,
\]
where $v_c = (0,\,1,\,-\tfrac{1}{3},\,\tfrac{1}{7},\ldots,\tfrac{(-1)^{m+1}}{2^m-1},\ldots)$ and
\[
P_{r,k}(x) = (1+2x)^r(1+x)^{k-r}.
\]

For $r=k$, we have
\[
P_{k,k}(x) = (1+2x)^k = \sum_{m=0}^{k} \binom{k}{m} (2x)^m,
\]
so its coefficient sequence is $(1,\,2\binom{k}{1},\,2^2\binom{k}{2},\ldots,2^k\binom{k}{k},0,\ldots)$.

Thus
\[
c(k,k) = \sum_{m=1}^{k} \binom{k}{m} 2^m \cdot \frac{(-1)^{m+1}}{2^m-1}.
\]

Similarly, for $r=0$,
\[
P_{0,k}(x) = (1+x)^k = \sum_{m=0}^{k} \binom{k}{m} x^m,
\]
so
\[
c(0,k) = \sum_{m=1}^{k} \binom{k}{m} \cdot \frac{(-1)^{m+1}}{2^m-1}.
\]

Subtracting gives
\[
c(k,k) - c(0,k) = \sum_{m=1}^{k} \binom{k}{m} (-1)^{m+1} \left( \frac{2^m}{2^m-1} - \frac{1}{2^m-1} \right)
= \sum_{m=1}^{k} \binom{k}{m} (-1)^{m+1} = 1,
\]
by the binomial identity $\sum_{m=0}^{k} \binom{k}{m} (-1)^m = 0$.


\item $c(r,k) = \frac{c(r,k-1) + c(r+1,k)}{2}$:

By definition,
\[
c(r,k) = \langle v_c, P_{r,k} \rangle,
\]
where $P_{r,k}(x) = (1+2x)^r(1+x)^{k-r}$.

From the polynomial identity \eqref{eq:2prk=prk+prk}:
\[
P_{r,k-1}(x) + P_{r+1,k}(x) = 2P_{r,k}(x).
\]

Using linearity of the inner product:
\begin{align*}
\langle v_c, P_{r,k-1}\rangle + \langle v_c, P_{r+1,k}\rangle
&= \langle v_c, 2P_{r,k}\rangle \\
&= 2\langle v_c, P_{r,k}\rangle.
\end{align*}

Thus,
\[
c(r,k-1) + c(r+1,k) = 2c(r,k),
\]
and dividing by $2$ gives
\[
c(r,k) = \frac{c(r,k-1) + c(r+1,k)}{2}.
\]

    
\end{enumerate}
\end{proof}
    


% Hence
% \[
% c(r,k) = \langle \sum q, P_{r,k} \rangle = 
% \]
% \[
% = \langle (0,\bigg(\frac{(-1)^{j-1}}{2^j-1}\bigg)^{\infty}_{j=1}), P_{r,k} \rangle.
% \]

% Pak se suma da vytknout
% \[
% c(r,k) = \sum\langle  q, P_{r,k} \rangle,
% \]


% Od tohoto je to uz jen krucek k $b_{r,k}$
\subsection{Corollaries}

As a corollary, a finite vector $v_c$ can be embedded as an infinite sequence $(q_j)$ in $\ell^{2}$, enabling methods that require infinite-dimensional representations.

\begin{definition}[Geometric vector $v_j$ and weight increment $q_j$]
\label{definition:vjqj}
Let $e_0=(1,0,0,\dots)$ denote the first standard basis vector of $\ell^{2}$.
For each integer $j\ge 1$ we define the geometric vector
\[
v_j := \bigl(1,\,-2^{-j},\,( -2^{-j})^{2},\,( -2^{-j})^{3},\ldots \bigr),
\]
that is,
\[
(v_j)_m = (-2^{-j})^{m} \qquad \text{for all } m\ge 0.
\]
We then define the associated weight increment
\[
q_j := e_0 - v_j,
\]
which has the explicit coordinate form
\[
q_j = \bigl(0,\,2^{-j},\,-2^{-2j},\,2^{-3j},\,-2^{-4j},\ldots\bigr),
\]
so that $(q_j)_0 = 0$ and
\[
(q_j)_m = (-1)^{m-1}2^{-jm},\qquad m\ge 1.
\]
\end{definition}


We now show that the vector $q_j$ belongs to the space $\ell^{p}$ by 
verifying that it satisfies the defining condition~\eqref{eq:l2iflesstheninfty}.

\begin{proof}
By Definition~\ref{definition:vjqj} we have
\[
q(j) = \bigl(0,\,2^{-j},\,-2^{-2j},\,2^{-3j},\,-2^{-4j},\ldots\bigr),
\]
so for every $m \ge 1$,
\[
q(j)_m = (-1)^{m-1} 2^{-jm}.
\]
Fix $m \ge 1$. Then
\[
\sum_{j=1}^{\infty} q(j)_m
 = \sum_{j=1}^{\infty} (-1)^{m-1} 2^{-jm}
 = (-1)^{m-1} \sum_{j=1}^{\infty} (2^{-m})^{j}.
\]
The inner sum is a geometric series with first term $a = 2^{-m}$ and ratio
$r = 2^{-m}$, where $0 < 2^{-m} < 1$. Hence
\[
\sum_{j=1}^{\infty} (2^{-m})^{j}
 = \frac{2^{-m}}{1-2^{-m}},
\]
and therefore
\[
\sum_{j=1}^{\infty} q(j)_m
 = (-1)^{m-1} \frac{2^{-m}}{1-2^{-m}}
 = \frac{(-1)^{m-1}}{2^{m}-1}.
\]
Since $q(j)_0 = 0$ for all $j$, the zeroth component of the sum is also $0$.
This proves the claim.
\end{proof}

\begin{proposition}
\label{prop:sumq}
The series of vectors $\sum_{j=1}^{\infty} q(j)$ converges in $\ell^{2}$ and its
sum is given componentwise by
\[
\sum_{j=1}^{\infty} q(j)
 = \left( 0,\;\biggl(\frac{(-1)^{k-1}}{2^{k}-1}\biggr)_{k=1}^{\infty} \right),
\]
\end{proposition}

\begin{proof}
We recall the formula for a geometric series
\[
S_n = a\,\frac{1-r^n}{1-r},
\qquad
S_\infty = \frac{a}{1-r}
\quad\text{for } |r|<1.
\]

Next, we write out the first few rows of the vectors $q(j)$:
\begin{align*}
q(1) &= (0,2^{-1},-2^{-2},\cdots,(-1)^{k+1}2^{-k},\cdots)\\
q(2) &= (0,2^{-2},-2^{-4},\cdots,(-1)^{k+1}2^{-2k},\cdots)
\end{align*}
and so on.  

For each fixed column index $k \ge 1$ we now look at the $k$-th entries
of these rows.
which satisfies $0<2^{-k}<1$.

Therefore, by the geometric series formula,
\begin{align}
&S_n = a\frac{1-r^n}{1-r}\\
      &S_\infty = (-1)^{k+1}2^{-k}\frac{1-(\frac{(-1)^{k+1}2^{-2k}}{(-1)^{k+1}2^{-k}})^\infty}{1-(\frac{(-1)^{k+1}2^{-2k}}{(-1)^{k+1}2^{-k}})} = (-1)^{k+1}2^{-k}\frac{1-(2^{-k})^\infty}{1-(2^{-k})}=\\
      &=\frac{(-1)^{k-1}}{2^k-1} = \sum_{j=1}^{\infty}q(j)
\end{align}
Since each $q(j)$ has zero in the first component, the zeroth component of the
sum is also $0$.  This yields
\[
\sum_{j=1}^{\infty} q(j)
 = \left( 0,\;\biggl(\frac{(-1)^{k-1}}{2^{k}-1}\biggr)_{k=1}^{\infty} \right),
\]
which is the desired result.
\end{proof}

Thus
\[
c(r,k)
 = \left\langle \sum_{j=1}^{\infty} q(j),\, P_{r,k} \right\rangle.
\]

We recall that the inner product in an inner product space is linear in its
first argument.  
In particular, axioms (IP1)–(IP2) in Kreyszig~\cite[p.~129]{kreyszig1991introductory}
give
\[
\langle x+y,\, z\rangle = \langle x, z\rangle + \langle y, z\rangle,
\qquad
\langle \alpha x,\, z\rangle = \alpha\,\langle x, z\rangle.
\]
We therefore use this linearity when computing 
\(\langle \sum_{j=1}^{\infty} q(j),\, P_{r,k} \rangle\).

Since $P_{r,k}$ has only finitely many nonzero components, linearity of the
inner product in $\ell^{2}$ yields
\[
c(r,k)
 = \left\langle \sum_{j=1}^{\infty} q(j),\, P_{r,k} \right\rangle
 = \sum_{j=1}^{\infty} \langle q(j),\, P_{r,k} \rangle.
\]

\subsection{Notable candidate definitions}

\begin{lemma}
For integers $0 \le r < k$, the quantity $c(r,k)$ admits the representation
\[
c(r,k) = \sum_{m=1}^{\infty} \Bigl[\,1 - (1-2^{1-m})^{r}(1-2^{-m})^{\,k-r}\Bigr].
\]
\end{lemma}

\begin{proof}
By definition in $\ell^{2}$,
\[
c(r,k) = \langle v_c, P_{r,k} \rangle,
\]
where $v_c$ is our weight vector $v_c=(0,1,-\frac{1}{3},\frac{1}{7},\dots,\frac{(-1)^{i+1}}{2^i+1},\dots)$ and $P_{r,k}(x) = (1+2x)^{r}(1+x)^{\,k-r}$.

Expanding the inner product gives
\[
c(r,k) = \sum_{j=1}^{k} (-1)^{j-1}\,\frac{1}{2^{j}-1}\,\alpha_j,
\]
with $\alpha_j$ the $j$-th coefficient of $P_{r,k}(x)$.

Using the decomposition
\[
\frac{1}{2^{j}-1} = \sum_{m=1}^{\infty} 2^{-jm},
\]
and interchanging sums:
\[
c(r,k) = \sum_{m=1}^{\infty} \sum_{j=1}^{k} (-1)^{j-1}\,\alpha_j\,2^{-jm}.
\]

By the generating function identity $\sum_{j \ge 0}\alpha_j a^j = P_{r,k}(a)$:
\[
\sum_{j=1}^{\infty} (-1)^{j-1}\,\alpha_j\,2^{-jm}
= 1 - P_{r,k}(-2^{-m})
= 1 - (1-2^{1-m})^{r}(1-2^{-m})^{\,k-r}.
\]

Summing over $m$ gives the result.
\end{proof}

\subsection{brk}

In what follows, we introduce a quantity that measures the change of
\(c(r,k)\) along the \(r\)-direction.  
It will play a central role in understanding how the values of \(c(r,k)\)
propagate across each column \(k\) and may lead to approximation.

\begin{definition}[\(b(r,k)\)]
For integers \(k \ge 0\) and \(1 \le r \le k\) we define
\[
b(r,k) := c(r,k) - c(r-1,k).
\]
\end{definition}

\begin{lemma}\label{lem:c-from-b}
With the convention that empty sums are zero.
For \(0 \le r \le k\) set
\begin{equation}\label{eq:def-c-from-b}
c(r,k) \;:=\; \sum_{j=1}^{k} b(1,j) \;+\; \sum_{i=1}^{r} b(i,k).
\end{equation}
Then \(c(r,k)\) satisfies all its previously stated equations.
\begin{align}
c(0,0) &= 0, \label{eq:c00-b}\\
c(k,k)-c(0,k) &= 1, \label{eq:ckk-c0k-b}\\
2c(r,k) &= c(r+1,k)+c(r,k-1), \qquad 0 \le r \le k-1,\ k\ge 1. \label{eq:crk-b}
\end{align}
\end{lemma}

\begin{proof}
We verify the three identities separately.

\medskip
\noindent\textbf{(i) The condition \eqref{eq:c00-b}.}
For \(k=r=0\) the definition \eqref{eq:def-c-from-b} gives
\[
c(0,0) = \sum_{j=1}^{0} b(1,j) + \sum_{i=1}^{0} b(i,0) = 0,
\]
so \eqref{eq:c00-b} holds.

\medskip
\noindent\textbf{(ii) The condition \eqref{eq:ckk-c0k-b}.}
For \(r=k\) we have
\[
c(k,k)
 = \sum_{j=1}^{k} b(1,j) + \sum_{i=1}^{k} b(i,k),
\qquad
c(0,k)
 = \sum_{j=1}^{k} b(1,j).
\]
Hence
\[
c(k,k) - c(0,k)
 = \sum_{i=1}^{k} b(i,k).
\]
By construction of \(b(i,k)\) (each column corresponding to a fixed \(k\))
we have
\[
\sum_{i=1}^{k} b(i,k) = 1,
\]
so \eqref{eq:ckk-c0k-b} follows.

\medskip
\noindent\textbf{(iii) The condition \eqref{eq:crk-b}.}

\textcolor{red}{TODO: POMOC TADY JSEM ZTRACEN}

\paragraph{Main identity.}
Recall the definition
\[
b(r,k) := c(r,k) - c(r-1,k),
\]
for integers \( k \ge 0 \) and \( 1 \le r \le k \).

We aim to prove the identity
\[
2c(r,k) = c(r+1,k) + c(r,k-1).
\]

First, observe that
\[
c(r+1,k) = b(r+1,k) + c(r,k),
\]
since \( b(r+1,k) = c(r+1,k) - c(r,k) \).
Similarly,
\[
c(r,k-1) = b(r,k-1) + c(r-1,k-1).
\]

Substituting these into the right-hand side of the desired identity gives
\begin{align*}
c(r+1,k) + c(r,k-1)
&= \bigl(b(r+1,k) + c(r,k)\bigr) + \bigl(b(r,k-1) + c(r-1,k-1)\bigr) \\
&= c(r,k) + b(r+1,k) + b(r,k-1) + c(r-1,k-1).
\end{align*}

Thus, the equation \( 2c(r,k) = c(r+1,k) + c(r,k-1) \) is equivalent to
\[
c(r,k) = b(r+1,k) + b(r,k-1) + c(r-1,k-1).
\]

Finally, note that the identity
\[
2P_{r,k} = P_{r+1,k} + P_{r,k-1}
\]
holds for the polynomials \( P_{r,k} \). Since \( c(r,k) \) is defined via a linear inner product with \( P_{r,k} \), linearity implies
\[
2c(r,k) = c(r+1,k) + c(r,k-1),
\]
which completes the proof.
\end{proof}


\subsection{Matrix identity to show that b(r,k) is a function}
\textcolor{red}{TODO:}

\subsection{using Prk to construct Brk}
\textcolor{red}{TODO:}
a jeste chceme ukazat ze z $Prk$u umime $b_{rk}$, ale to je obdobne jako ze z $Prk$u umime $c_{rk}$ to uz mame nad timto


\section{naznak aproximace brk?}

We start with an approximation of $b_{r,k}$ like:
$$
b_{rk}=\frac{1}{H_{2k-2}-H_{k-2}}\cdot\frac{1}{r+k-2}
$$
First being well-known function to approximate, yielding
$$
b_{rk}\approx\frac{1}{ln(2)}\cdot\frac{1}{r+k-2}
$$

We substitute this approximation into all the lemmas equations, yelding for the first equation
$$
\sum_{i=1}^{k}b_{ik} \approx\frac{1}{ln(2)}(\frac{1}{k-1}+\frac{1}{k}+\frac{1}{k+1}+\cdots+\frac{1}{2k-2})\approx\frac{H_{2k-2}-H_{k-2}}{ln(2)}\approx \frac{ln(2)}{ln(2)} \approx 1.
$$
this is consistent with our assumption.

Substituting into the second equation yields
\begin{align*}
\frac{1}{2}b_{r+1k}+\frac{1}{2}b_{rk-1}&\approx\frac{1}{ln2}\frac{1}{2}(\frac{1}{r+k-3}+\frac{1}{r+k-1})\approx\\
&\approx\frac{1}{ln2}\frac{1}{2}\frac{r+k-1+r+k-3}{(r+k-3)(r+k-1)}\approx\frac{1}{ln2}\frac{r+k-2}{(r+k-3)(r+k-1)}.
\end{align*}


We check whether this result provides a sufficiently good approximation to the previously obtained value ($b_{rk}=\frac{1}{ln(2)}\cdot\frac{1}{r+k-2}$).

We label the functions:
$$
f(r,k) = \frac{1}{ln(2)}\cdot\frac{1}{r+k-2}
$$
and
$$
g(r,k)=\frac{1}{ln2}\frac{r+k-2}{(r+k-3)(r+k-1)}.
$$

\subsection{Theory}
\begin{definition}[Order of Magnitude, p.~19]
It is said that $f(\varepsilon) = O(q(\varepsilon))$ for $\varepsilon \to 0$, if
\[
\lim_{\varepsilon \to 0} \frac{f(\varepsilon)}{q(\varepsilon)} = A, \quad 0 < |A| < \infty.
\]
\end{definition}

In addition, it is said that the functions $f(\varepsilon)$ and $q(\varepsilon)$ have the same \emph{order of magnitude}. 
strana 19 v Gao:
\url{https://utbcz-my.sharepoint.com/:b:/g/personal/a_ulrich_utb_cz/IQCowD5Uz9tnQ4FZdzpTsfLLAYpgJpNYeJcZxR5YsaERZTI?e=BlmlFg}.


Furthermore, de Bruijn mentions in Chapter~1.4 (\emph{Asymptotic equivalence}, p.~11) that:

\begin{quote}
We say that $f(x)$ and $g(x)$ are asymptotically equivalent as $x \to \infty$, if the quotient $f(x)/g(x)$ tends to unity. Our notation is
\[
f(x) \sim g(x) \qquad (x \to \infty).
\]
\end{quote}

Alternatively, this can be expressed as:
\[
\lim_{x \to \infty} \frac{f(x)}{g(x)} = 1.
\]


This notation will also be used for all other ways of passing to a limit.


\begin{align*}
&\lim_{k\rightarrow\infty}\frac{f}{g} =\lim_{k\rightarrow\infty} \frac{\frac{1}{ln(2)}\cdot\frac{1}{r+k-2}}{\frac{1}{ln2}\frac{r+k-2}{(r+k-3)(r+k-1)}}=\\
&=\lim_{k\rightarrow\infty}\frac{(r+k-3)(r+k-1)}{(r+k-2)^2}=[\frac{(r^2+rk-r+kr+k^2-k-3r-3k+3)}{(r+k)^2-2(r+k)2+4}]=\\
&=[\frac{r^2+rk-r+kr+k^2-k-3r-3k+3}{r^2+2rk+k^2-4r-4k+4}=\lim_{k\rightarrow\infty}[\frac{k^2(1+...)}{k^2(1+...)}]=1    
\end{align*}

Thus, we have shown that $f(r,k)$ and $g(r,k)$ are asymptotically equivalent as $k \to \infty$, i.e.,
\[
f(r,k) \sim g(r,k) \qquad (k \to \infty),
\]
in the sense of de Bruijn~\cite[Ch.~1.4]{debruijn1981asymptotic}, since their quotient tends to unity.

So we have shown that the functions $f(r,k)$ and $g(r,k)$ have the same order of magnitude.

To confirm that also the actual values agree (at least up to a certain order \( n \)), we refer to Bender's concept concerning the comparison of two functions.

\paragraph{Asymptotic Power Series (Bender \& Orszag, p.~89,\cite{bender2013advanced})}
\url{https://utbcz-my.sharepoint.com/:b:/g/personal/a_ulrich_utb_cz/IQBKSQxcCYLyT7T8GdrWJyPHAVaVW0szQgG_BqGHQRLsQ10?e=7FC76l}
In Chapter 3, Bender and Orszag define an asymptotic power series as follows:
\begin{quote}
The power series $\sum_{n=0}^\infty a_n (x-x_0)^n$ is said to be asymptotic to the function $y(x)$ as $x \to x_0$ if
\[
y(x) - \sum_{n=0}^N a_n (x-x_0)^n \ll (x-x_0)^N \qquad (x \to x_0)
\]
for every $N$.
\end{quote}
\textit{In words: The remainder after $N$ terms is much smaller than the last retained term as $x \to x_0$.}

They also give an equivalent formulation:
\begin{quote}
\[
y(x) - \sum_{n=0}^M a_n (x-x_0)^n \sim a_M (x-x_0)^M \qquad (x \to x_0)
\]
\end{quote}
\textit{In words: If you subtract the first $M$ terms, the remainder is of the same order as the next term.}

In particular, if two functions $f$ and $g$ have asymptotic expansions that agree up to order $n$, then their difference satisfies $f(x) - g(x) = O((x-x_0)^{n+1})$ as $x \to x_0$.

Thus, if we wish to demonstrate that two functions agree up to order $n$, it suffices to show that their difference is $O((x-x_0)^{n+1})$ as $x \to x_0$.

\subsubsection{Taylor series and limit approximation of f and g}
First, we start with previously defined approximations $f(r,k)$ and $g(r,k)$.
Since the following approximation methods are better used with one variable, we substitute:
$$
x :=r+k-2,
$$
yielding
\begin{align*}
    F(x) &= \frac{1}{ln(2)}\frac{1}{x}\\
    G(x) &= \frac{1}{ln(2)}\frac{x}{x^2-1}
\end{align*}

Then we subtract these to show the the maximum error that can occur between these two approximations.
$$
R(x)=G(x)-F(x)=\frac{1}{ln(2)}(\frac{x}{x^2-1}-\frac{1}{x})=\frac{1}{ln(2)}\frac{1}{x(x^2-1)}
$$

We substitute $x\rightarrow t$, yielding

\begin{align*}
    R(t) = G(t) - F(t)= \frac{1}{ln(2)}\biggr(\frac{1}{\frac{1}{t}(\frac{1}{t^2}-1)}\biggl)= \frac{1}{ln(2)}\biggr(\frac{t^3}{1-t^2}\biggl)
\end{align*}
Now we do Taylor expansion up to third derivative, yielding:
\begin{align*}
T(t)&=0+0t+0t^2+\frac{6t^3}{3!}+O(t^4)\\    
T(x)&=0+0\frac{1}{t}+0\frac{1}{x^2}+\frac{6}{3!}\frac{1}{x^3}+O(\frac{1}{x^4})
\end{align*}
Hence these functions agree up to order of $2$.

We try different approach now (Spivak, page 418).
\begin{quote}
Following Spivak~\cite{SpivakMichael1994C}, two functions \( f \) and \( g \) are said to be \emph{equal up to order \( n \) at \( a \)} if
\[
\lim_{x \to a} \frac{f(x) - g(x)}{(x - a)^n} = 0.
\]
\end{quote}

Using this approach with our functions $G(t)$ and $F(t)$ yields:

\begin{align*}
    2.)\lim_{x\rightarrow0^+}\frac{G(t)-F(t)}{t^2} &=0\\
    3.)\lim_{x\rightarrow0^+}\frac{G(t)-F(t)}{t^3} &= \lim_{x\rightarrow0^+}\frac{t^3}{t^3(1-t^2)}=1
\end{align*}
Since the last $n$ for which the limit was zero was $n = 2$, the functions are equal up to order $2$ at $0$.



\section{jak se lisi vysledky crk s missing a ck bez missing}
- plus mozna i dukaz opet ze to opravdu bude fungovat vzdy

\section{priklad pro dusana}

\section{zamysleni - jak by slo crk aproximovat - future research}


\bibliographystyle{alpha}
\bibliography{bibli}

\end{document}
